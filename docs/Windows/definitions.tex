\usepackage{listings,setspace}


\usepackage{tikz}
\usetikzlibrary{calc,through,arrows,backgrounds,shapes,fit,positioning,shadows,decorations.text,,decorations.shapes,trees}
%\usepackage{fontspec}
%\setmainfont[Mapping=tex-text]{Times New Roman}

%\newfontfamily\listingsfont[Scale=0.8]{Courier} 
%\newfontfamily\listingsfontinline[Scale=0.8]{Courier New} 
\usepackage{color}
\definecolor{sh_comment}{rgb}{0.12, 0.38, 0.18 } %adjusted, in Eclipse: {0.25, 0.42, 0.30 } = #3F6A4D
\definecolor{sh_keyword}{rgb}{0.80, 0.08, 0.25}  % #5F1441
\definecolor{sh_string}{rgb}{0.06, 0.10, 0.98} % #101AF9



\usepackage{times}
\lstdefinelanguage{sdml}
{
morekeywords={true,false,class,Int,Boolean,Long,List,Map,String,Set,find_package,cmake_minimum_required,if,endif,link_directories,link_libraries,include_directories,add_executable,find_library,mark_as_advanced,add_definitions,find_program,message,option,else,unset,install},
sensitive=false,
morecomment=[l]{\#},
morecomment=[s]{/*}{*/},
morestring=[b]",
% basicstyle=\listingsfont,
 rulesepcolor=\color{black},
 showspaces=false,
 showstringspaces=false,
 showtabs=false,tabsize=2,
  basicstyle= \footnotesize\ttfamily,
  stringstyle=\color{sh_string},
  keywordstyle = \color{sh_keyword}\bfseries,
  commentstyle=\color{sh_comment}\itshape,
%keywordstyle=\color{purple}\it\bfseries,
%commentstyle=\color{blue}\it,
%backgroundcolor=\color{blue!4},
escapeinside={?}{?},
}
\lstset{language=sdml}


\tikzfading[name=fade inside, inner color=transparent!80,
outer color=transparent!30]


\tikzset{thick,  draw=blue!50, netnode/.style={scale=0.8, 
		  shape=circle,
		  % The border:
                   thick,
                   draw=blue!80,
                   % The filling:
                   top color=white,              % a shading that is white at the top...
                   bottom color=blue!50!white!20, % and something else at the bottom
                   % Font
                   font=\itshape,
		  drop shadow},
netnodeB/.style={scale=0.8, 
		  shape=circle,
		  % The border:
                   thick,
                   draw=blue!80,
                   % The filling:
                   top color=white,              % a shading that is white at the top...
                   bottom color=blue!50!white!20, % and something else at the bottom
                   % Font
                   font=\itshape},
latnode/.style={draw,circle, color=blue}, 
latedge/.style={very thick, color=blue}
}


\newcommand{\declaration}[1]{
% \begin{center}
% \centering
{\begin{tikzpicture}[depM/.style={scale=1, 
		  shape=rectangle,
		  % The border:
                   thick,
                   draw=green!1,
                   % The filling:
                   top color=white,              % a shading that is white at the top...
                   bottom color=green!40!white!20, % and something else at the bottom
                   % Font
                   font=\itshape,
		  drop shadow}]
\draw node[depM]{{#1}}; 
\end{tikzpicture}
% \end{center}
}}

\newcommand{\query}[1]{
% \begin{center}
\begin{tikzpicture}[depM/.style={scale=1, 
		  shape=rectangle,
		  % The border:
                   thick,
                   draw=red!1,
                   % The filling:
                   top color=white,              % a shading that is white at the top...
                   bottom color=red!50!white!20, % and something else at the bottom
                   % Font
                   font=\itshape,
		  drop shadow}]
\draw node[depM]{{#1}}; 
\end{tikzpicture}
% \end{center}
}

\usepackage{framed}
\usetikzlibrary{decorations.pathmorphing,calc}
\pgfmathsetseed{1} % To have predictable results
% Define a background layer, in which the parchment shape is drawn
\pgfdeclarelayer{background}
\pgfsetlayers{background,main}

% define styles for the normal border and the torn border
\tikzset{
  normal border/.style={shading=axis, top color=white!70, bottom color=red!10, decorate, color=black!20, 
     decoration={random steps, segment length=2.5cm, amplitude=.7mm}},
  torn border/.style={orange!30!black!5, decorate, 
     decoration={random steps, segment length=.3cm, amplitude=.5mm}
},
  code border/.style={shading=axis, top color=white!70, bottom color=blue!10, decorate, color=black!20},
  torncode border/.style={orange!30!black!5, decorate   
}
}

% Macro to draw the shape behind the text, when it fits completly in the
% page
\def\parchmentframe#1{
\tikz{
  \node[inner sep=1em] (A) {#1};  % Draw the text of the node
  \begin{pgfonlayer}{background}  % Draw the shape behind
  \fill[normal border] 
        (A.south east) -- (A.south west) -- 
        (A.north west) -- (A.north east) -- cycle;
  \end{pgfonlayer}}}

% Macro to draw the shape, when the text will continue in next page
\def\parchmentframetop#1{
\tikz{
  \node[inner sep=1em] (A) {#1};    % Draw the text of the node
  \begin{pgfonlayer}{background}    
  \fill[normal border]              % Draw the ``complete shape'' behind
        (A.south east) -- (A.south west) -- 
        (A.north west) -- (A.north east) -- cycle;
  \fill[torn border]                % Add the torn lower border
        ($(A.south east)-(0,.1)$) -- ($(A.south west)-(0,.1)$) -- 
        ($(A.south west)+(0,.1)$) -- ($(A.south east)+(0,.1)$) -- cycle;
  \end{pgfonlayer}}}

% Macro to draw the shape, when the text continues from previous page
\def\parchmentframebottom#1{
\tikz{
  \node[inner sep=1em] (A) {#1};   % Draw the text of the node
  \begin{pgfonlayer}{background}   
  \fill[normal border]             % Draw the ``complete shape'' behind
        (A.south east) -- (A.south west) -- 
        (A.north west) -- (A.north east) -- cycle;
  \fill[torn border]               % Add the torn upper border
        ($(A.north east)-(0,.2)$) -- ($(A.north west)-(0,.2)$) -- 
        ($(A.north west)+(0,.2)$) -- ($(A.north east)+(0,.2)$) -- cycle;
  \end{pgfonlayer}}}

% Macro to draw the shape, when both the text continues from previous page
% and it will continue in next page
\def\parchmentframemiddle#1{
\tikz{
  \node[inner sep=1em] (A) {#1};   % Draw the text of the node
  \begin{pgfonlayer}{background}   
  \fill[normal border]             % Draw the ``complete shape'' behind
        (A.south east) -- (A.south west) -- 
        (A.north west) -- (A.north east) -- cycle;
  \fill[torn border]               % Add the torn lower border
        ($(A.south east)-(0,.2)$) -- ($(A.south west)-(0,.2)$) -- 
        ($(A.south west)+(0,.2)$) -- ($(A.south east)+(0,.2)$) -- cycle;
  \fill[torn border]               % Add the torn upper border
        ($(A.north east)-(0,.2)$) -- ($(A.north west)-(0,.2)$) -- 
        ($(A.north west)+(0,.2)$) -- ($(A.north east)+(0,.2)$) -- cycle;
  \end{pgfonlayer}}}

% Define the environment which puts the frame
% In this case, the environment also accepts an argument with an optional
% title (which defaults to ``Example'', which is typeset in a box overlaid
% on the top border
\newenvironment{parchment}[1][Example]{%
  \def\FrameCommand{\parchmentframe}%
  \def\FirstFrameCommand{\parchmentframetop}%
  \def\LastFrameCommand{\parchmentframebottom}%
  \def\MidFrameCommand{\parchmentframemiddle}%
  \vskip\baselineskip
  \MakeFramed {\FrameRestore}
  \noindent\tikz\node[inner sep=1ex, draw=black!20,  fill=white, 
          anchor=west, overlay] at (0em, 1em) {\sffamily#1};\par}%
{\endMakeFramed}

% Macro to draw the shape behind the text, when it fits completly in the
% page
\def\codeparchmentframe#1{
\tikz{
  \node[inner sep=1em] (A) {#1};  % Draw the text of the node
  \begin{pgfonlayer}{background}  % Draw the shape behind
  \fill[code border] 
        (A.south east) -- (A.south west) -- 
        (A.north west) -- (A.north east) -- cycle;
  \end{pgfonlayer}}}

% Macro to draw the shape, when the text will continue in next page
\def\codeparchmentframetop#1{
\tikz{
  \node[inner sep=1em] (A) {#1};    % Draw the text of the node
  \begin{pgfonlayer}{background}    
  \fill[code border]              % Draw the ``complete shape'' behind
        (A.south east) -- (A.south west) -- 
        (A.north west) -- (A.north east) -- cycle;
  \fill[torncode border]                % Add the torn lower border
        ($(A.south east)-(0,.1)$) -- ($(A.south west)-(0,.1)$) -- 
        ($(A.south west)+(0,.1)$) -- ($(A.south east)+(0,.1)$) -- cycle;
  \end{pgfonlayer}}}

% Macro to draw the shape, when the text continues from previous page
\def\codeparchmentframebottom#1{
\tikz{
  \node[inner sep=1em] (A) {#1};   % Draw the text of the node
  \begin{pgfonlayer}{background}   
  \fill[code border]             % Draw the ``complete shape'' behind
        (A.south east) -- (A.south west) -- 
        (A.north west) -- (A.north east) -- cycle;
  \fill[torncode border]               % Add the torn upper border
        ($(A.north east)-(0,.2)$) -- ($(A.north west)-(0,.2)$) -- 
        ($(A.north west)+(0,.2)$) -- ($(A.north east)+(0,.2)$) -- cycle;
  \end{pgfonlayer}}}

% Macro to draw the shape, when both the text continues from previous page
% and it will continue in next page
\def\codeparchmentframemiddle#1{
\tikz{
  \node[inner sep=1em] (A) {#1};   % Draw the text of the node
  \begin{pgfonlayer}{background}   
  \fill[code border]             % Draw the ``complete shape'' behind
        (A.south east) -- (A.south west) -- 
        (A.north west) -- (A.north east) -- cycle;
  \fill[torncode border]               % Add the torn lower border
        ($(A.south east)-(0,.2)$) -- ($(A.south west)-(0,.2)$) -- 
        ($(A.south west)+(0,.2)$) -- ($(A.south east)+(0,.2)$) -- cycle;
  \fill[torncode border]               % Add the torn upper border
        ($(A.north east)-(0,.2)$) -- ($(A.north west)-(0,.2)$) -- 
        ($(A.north west)+(0,.2)$) -- ($(A.north east)+(0,.2)$) -- cycle;
  \end{pgfonlayer}}}


\newenvironment{codeparchment}[1][Example]{%
  \def\FrameCommand{\codeparchmentframe}%
  \def\FirstFrameCommand{\codeparchmentframetop}%
  \def\LastFrameCommand{\codeparchmentframebottom}%
  \def\MidFrameCommand{\codeparchmentframemiddle}%
  \vskip\baselineskip
  \MakeFramed {\FrameRestore}
  \noindent\tikz\node[inner sep=1ex, draw=black!20,  fill=white, 
          anchor=west, overlay] at (0em, 1em) {\sffamily#1};\par}%
{\endMakeFramed}

\lstnewenvironment{sdmlcode}
{\singlespacing\lstset{language=sdml}}
{}

